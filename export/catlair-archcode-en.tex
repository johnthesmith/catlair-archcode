\documentclass[final]{article}
\usepackage[T2A]{fontenc} % Cyrillic support
\usepackage[utf8]{inputenc} % UTF-8 encoding
\usepackage{amsmath} % Mathematical symbols
\usepackage{amssymb} % Additional mathematical symbols
\usepackage{graphicx}
\usepackage{hyperref}
\usepackage{ulem}
\usepackage{float}
\usepackage{lastpage}

\hypersetup{
    colorlinks=true,          % Enables link colors
    linkcolor=black,          % Link color (black)
    urlcolor=black,           % URL color (black)
}

\begin{document}

    \title{Architectural Code in the Catlair Entity Model}
    \author{
        Ruslan Cherkas
        Igor Cheledinov
    }
    \date{2025-04-19}

    \begin{small}
        \begingroup
        \renewcommand{\baselinestretch}{0.8}
        
        \renewcommand{\contentsname}{Contents}
        \maketitle
        \tableofcontents
        
        \section*{Notes}
            \begin{enumerate}
                \item https://doi.org/10.5281/zenodo.15250894
                \url{https://doi.org/https://doi.org/10.5281/zenodo.15250894}
                
                \item Permanent document link [ru]:
                \url{https://github.com/johnthesmith/catlair-archcode/blob/main/export/catlair-archcode-ru.pdf}

                \item Permanent document link [en]:
                \url{https://github.com/johnthesmith/catlair-archcode/blob/main/export/catlair-archcode-en.pdf}

                \item The article is published under the license:
                \url{https://creativecommons.org/licenses/by-sa/4.0/} CC BY-SA 4.0
            \end{enumerate}
        \endgroup
    \end{small}


    \section{Introduction}

        This article presents a concept for describing the architecture of information systems
        within a minimal set of rules.

        The concept is based on the use of an entity hierarchy. The description is universal and minimalist.
        It includes: an entity index, context-dependent properties, and relationships.

    \section{Definitions}

        The concept operates with the following specific terms:

        \begin{enumerate}

            \item Entity — a named abstraction representing an object,
            concept, role, or any other phenomenon from the real world in
            an information system. Entities form hierarchies, can have
            properties, and can be interconnected.

            In the text, the term entity is denoted as \texttt{entity}, and
            the set of entities is denoted as \texttt{entities}.

            \item Entity property — a description of the entity that defines its
            characteristics and features. An example of an entity property in
            architecture might include a name, description, quantitative indicators
            such as volume, etc.

            In the text, the term property is denoted as \texttt{prop}, and the set
            of properties is denoted as \texttt{props}.

            \item Entity relationships — a means of defining dependencies or
            interactions between entities.

            In the text, relationships are denoted as \texttt{link}, and the set
            of relationships is denoted as \texttt{links}.

            \item Contexts — views on the properties and relationships of entities
            that are specific to different participants and situations, depending on the environment.
            An example of a context could be the description language, differences in
            perception of the entity from the business logic side and the developer's side, and the level
            of detail in the diagrams.

            In the text, the term context is denoted as \texttt{context}, and the set
            of contexts is denoted as \texttt{contexts}.

        \end{enumerate}

    \section{Theory}

        The concept is based on the following principles:

        \begin{enumerate}
            \item Any concept in an information system should be described as
            an entity.
            
            \item For each entity, it is necessary to minimally define:
            \begin{itemize}
                \item the fact of existence;
                \item the type of the entity.
            \end{itemize}

            \item All other descriptions and properties of the entity should be considered secondary and context-dependent, namely:
            \begin{itemize}
                \item the relationship between entities;
                \item the attributes or properties of the entity.
            \end{itemize}

            \item The concept adheres to the
            \href{https://en.wikipedia.org/wiki/Separation_of_concerns}{principle of
            separation of concerns}, and therefore separates the description of entities and their
            relationships from the implementation of actions with them and their
            interpretation for specific purposes.

        \end{enumerate}

    \section{Practice}

        For practical examples, YAML syntax is used.

        \subsection{First Entity}

            \begin{enumerate}

                \item Entities use a tuple relationship in key-value format, where the key is the entity identifier, and the value is the identifier of its type.

                This corresponds to a relational representation with two attributes:

                \begin{itemize}
                    \item \texttt{id} - the entity identifier;
                    \item \texttt{type} - the identifier of the entity type (which, in turn, is also an entity).
                \end{itemize}

                \item Since the entity type is mandatory, and no type exists yet, the first entity should self-type in the \texttt{entities} collection:

                \begin{verbatim}
                entities:
                    entity: "entity"
                \end{verbatim}


                \textit{Note:} The self-typing tuple creates a new domain. Their number is unlimited, and each can develop independently.

            \end{enumerate}

    \subsection{Model Development}

        \begin{enumerate}
            \item Let's define several entities that will be used later in the description of the architecture.

            \begin{verbatim}
            entities:

                # Basic architecture entities

                # Component (services, state storages, etc...)
                component: "entity"
                # Service as a component
                service: "component"
                # Database as a component
                db: "component"
                # Agent capable of acting
                agent: "entity"
                # We defined the user as an agent
                user: "agent"
                # We defined the client as a user
                client: "user"

                # Let's add several architecture-specific entities:

                # Define the backend service
                my-backend: "service"
                # Define the database
                my-db: "db"
                # Define internal user Alice
                alice: "user"
                # Define client Bob
                bob: "client"

            \end{verbatim}

            \item Thus, through successive declarations, diverse entities have been defined, which will be used later.

        \end{enumerate}

    \subsection{Contexts}

        \begin{enumerate}

            \item The description of an entity differs for participants depending on the environment. Such differences are defined by the context.

            \item A simple example is the description of an entity in a human-readable form in different languages.

            \item A context may not only be a language, but also any specific perspective on entities from various groups, departments, divisions, or tasks.

            \item Let's add several contexts.

            \begin{verbatim}
            entities:

                # Context as an entity
                context: "entity"

                # Language as a context
                lang: "context"
                # To-do plans as a context
                todo: "context"
                # As-is description as a context
                asis: "context"

                # Russian language
                ru: "lang"
                # English language
                en: "lang"

            \end{verbatim}

        \end{enumerate}

    \subsection{Relations}
        \begin{enumerate}

            \item After adding entities, their interconnections should be considered.
            This is done using the "links" section.

            \item Relations are described as a typed direction from one entity 
            to another, where the type is also an entity.

            \item For relations, it is possible to specify multiple contexts 
            that allow representing the relation for different situations. The 
            context is optional. If no context is specified, it is interpreted 
            at the presentation level.

            \item In general, relations can be defined as a tuple of attributes, 
            each of which can contain one or more entities:

            \begin{itemize}
                \item \texttt{from} - the source entity of the relation;
                \item \texttt{to} - the direction entity of the relation;
                \item \texttt{type} - the type of the relation entity;
                \item \texttt{context} - an optional list of contexts for which the relation is relevant;
                \item \texttt{tuple} - an optional list of specific attributes for the relation in key-value format;
            \end{itemize}

            \item Let's describe some relations between entities:

            \begin{verbatim}

            entities:
                # Relation as an entity
                link: "entity"

            # Define relations between entities in the "links" section
            links:
                -
                    # Define that the service connects to the database
                    from: "my-backend"
                    to: "my-db"
                    link: "connect"
                -
                    # Define that the client connects to the service
                    from: "client"
                    to: "service"
                    link: "connect"
                -
                    # Alice can read and insert data
                    from: "alice"
                    link:
                        - "select"
                        - "insert"
                    to: "my-db"
                    context: "right"
                -

                    # Bob has select rights for the service in the "right" context
                    # for the "asis" concept
                    from: "bob"
                    to: "service"
                    link: "select"
                    context:
                        - "right"
                        - "asis"
                    # It is planned that Bob will have creation and insertion rights
                    # according to the "todo" concept
                    from: "bob"
                    to: "service"
                    link:
                        - "select"
                        - "insert"
                    context:
                        - "right"
                        - "todo"

            \end{verbatim}

            \item This method of description can include many different 
            dependencies, including technical connections of components, 
            hierarchical subordination structures, component placement 
            locations, and more.

        \end{enumerate}


    \subsection{Entity Descriptions}
        \begin{enumerate}

            \item Further, entities can be described with specific properties, also in the context of contexts.

            \item In general, the properties of entities can be defined as a tuple of attributes:

            \begin{itemize}
                \item \texttt{entity} - the entity for which the description is made, it can contain many entities;
                \item \texttt{tuple} - a list of properties of the entity in key-value format;
                \item \texttt{context} - an optional list of contexts for which the description is made;
            \end{itemize}

            \item Descriptions of entities are made in the "props" section:

            \begin{verbatim}
            props:
                # Human-readable description for an entity
                -
                    entity: "entity"
                    context: "ru"
                    tuple:
                        name:"Entity"
                # Description of Alice outside of context
                -
                    entity: "alice"
                    tuple:
                        age:21
                        weight:71
                # Descriptions of properties for Alice and Bob in different languages
                -
                    entity: "alice"
                    context: "ru"
                    tuple:
                        name:"Алиса"
                -
                    entity: "alice"
                    context: "en"
                    tuple:
                        name:"Alice"
                -
                    entity: "bob"
                    context: "ru"
                    tuple:
                        name:"Боб"
                -
                    entity: "bob"
                    context: "en"
                    tuple:
                        name:"Bob"

            \end{verbatim}

            \item Similarly, any properties of entities can be described.

        \end{enumerate}


    \section{Applicability}

        \begin{enumerate}

            \item The examples provided demonstrate the applicability of the Catlair entity model for describing architectural models.

            \item The Catlair notation allows for a standardized description of the list of objects, components, and their interrelationships. Additionally, all of these can be represented from different perspectives, including chronological ones.

            \item The model is applicable for describing organizational structures, access rights, and technical components.

            \item The model can be used for the uniform generation of ER, BPMN, and C4 diagrams at all levels, while maintaining consistency in notation, with the format of presentation determined by the context.

        \end{enumerate}


    \section{Appendix}

        The appendix contains a compact listing of the examples described above 
        for general understanding.

        \begin{verbatim}
        entities:
            entity: "entity"
            component: "entity"
            service: "component"
            db: "component"
            agent: "entity"
            user: "agent"
            client: "user"
            my-backend: "service"
            my-db: "db"
            alice: "user"
            bob: "client"
            link: "entity"
            context: "entity"
            lang: "context"
            todo: "context"
            asis: "context"
            ru: "lang"
            en: "lang"
        links:         
            -              
                from: "my-backend"
                to: "my-db"
                link: "connect"
            -
                from: "client"
                to: "service"
                link: "connect"
            -              
                from: "alice"
                link:
                    - "select"
                    - "insert"
                to: "my-db"
                context: "right"
            - 
                from: "bob" 
                to: "service"
                link: "select"
                context: 
                    - "right"
                    - "asis"
                from: "bob" 
                to: "service"
                link:
                    - "select"
                    - "insert"
                context: 
                    - "right"
                    - "todo"
            props:
                -
                    entity: "entity"
                    context: "ru"
                    tuple: 
                        name:"Сущность"
                -
                    entity: "alice"
                    tuple: 
                        age:21
                        weight:71
                -
                    entity: "alice"
                    context: "ru"
                    tuple: 
                        name:"Алиса"
                -
                    entity: "alice"
                    context: "en"
                    tuple: 
                        name:"Alice"
                -
                    entity: "bob"
                    context: "ru"
                    tuple: 
                        name:"Боб"
                -
                    entity: "bob"
                    context: "en"
                    tuple: 
                        name:"Bob"
        \end{verbatim}

    \renewcommand{\refname}{References}
    \begin{thebibliography}{9}

        \bibitem{EWD447} EWD447
        \textit{On the role of scientific thought}
        \url{https://www.cs.utexas.edu/~EWD/transcriptions/EWD04xx/EWD447.html}

        \bibitem{SimonBrown}
        Simon Brown.\\
        \textit{Software Architecture as Code}.\\
        \url{https://static.codingthearchitecture.com/presentations/saturn2015-software-architecture-as-code.pdf}

        \bibitem{github}
        Cherkas R. Cheledinov I.\\
        \textit{Архитектура сущностей Catlair}.\\
        \url{https://github.com/johnthesmith/scraps/blob/main/ru/entities.md}

        \bibitem{RichardsFord}
        Mark Richards, Neal Ford.\\
        \textit{Is "Architecture as Code" the Future of Software Design?}\\
        \url{https://www.architectureandgovernance.com/applications-technology/is-architecture-as-code-the-future-of-software-design/}

    \end{thebibliography}

\end{document}


