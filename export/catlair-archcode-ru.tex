\documentclass[final]{article}
\usepackage[T2A]{fontenc} % Поддержка кириллицы
\usepackage[utf8]{inputenc} % Кодировка UTF-8
\usepackage{amsmath} % Математические символы
\usepackage{amssymb} % Дополнительные математические символы
\usepackage{graphicx}
\usepackage{hyperref}
\usepackage{ulem}
\usepackage{float}
\usepackage{lastpage}

\hypersetup{
    colorlinks=true,          % Включает цвета ссылок
    linkcolor=black,          % Цвет ссылок (синие)
    urlcolor=black,           % Цвет URL (синие)
}


\begin{document}

    \title{Архитектурный код в моделе сущностей Catlair}
    \author{
        Черкас Руслан
        Челединов Игорь
    }
    \date{2025-04-19}

    \begin{small}
        \begingroup
        \renewcommand{\baselinestretch}{0.8}

        \renewcommand{\contentsname}{Содержание}
        \maketitle
        \tableofcontents

        \section*{Примечания}
            \begin{enumerate}
                \item doi:10.5281/zenodo.14319493
                \url{https://doi.org/***}

                \item Постоянный адрес документа [ru]: 
                \url{https://github.com/johnthesmith/catlair-archcode/blob/main/export/catlair-archcode-ru.pdf}

                \item Постоянный адрес документа [en]: 
                \url{https://github.com/johnthesmith/catlair-archcode/blob/main/export/catlair-archcode-en.pdf}

                \item Статья опубликована под лицензией: 
                \url{https://creativecommons.org/licenses/by-sa/4.0/} CC BY-SA 4.0
            \end{enumerate}
        \endgroup
    \end{small}

    \section{Введение}

        Статья предлагает концепцию описания архитектуры информационных систем в 
        рамках минимального набора правил.

        Концепция основана на использовании иерархии сущностей. Описание 
        универсально и минималистично. Влючает: индекс сущностей, 
        контекстно-зависимые свойства и связи.

    \section{Определения} 

        Концепция оперирует следующими специфическими понятиями:

        \begin{enumerate}

            \item Сущность — именованная абстракция, отображающая объект, 
            понятие, роль или любое другое явление реального мира в 
            информационной системе. Сущности образуют иерархии, могут иметь 
            свойства и связываться друг с другом.

            Далее по тексту сущность обозначается как \texttt{entity}, а 
            множество сущностей — как \texttt{entities}.

            \item Свойство сущности — описание сущности, определяещее его 
            особенности, характеристики. Примером свойства сущности в 
            архитектуре можно назвать наименование, описание, количественные 
            показатели объема и тд.

            Далее по тексту свойства обозначается как \texttt{prop}, а множество 
            атрибутов — как \texttt{props}.

            \item Связи сущностей — средство определения зависимости или
            взаимодействия между сущностями. 

            Далее по тексту связи как \texttt{link}, а множество 
            связей — как \texttt{links}.

            \item Контексты — специфические для различных участников и ситуаций 
            взгляды на свойства и связи сущностей, в зависимости от окружения. 
            Примером контекста может выступать язык описания, различия в 
            восприятии сущности со стороны бизнес логики и разработчика, уровень 
            детализации схем.

            Далее по тексту контекст обозначается как \texttt{context}, а их 
            множество как \texttt{contexts}.

        \end{enumerate}

    \section{Теория}

        В основе концепции лежат следующие принципы:

        \begin{enumerate}
            \item Любое понятие в информационной системе следует описывать как 
            сущность.

            \item Для каждой сущности неоходимо минимально определить:
            \begin{itemize}
                \item факт существования;
                \item тип сущности.
            \end{itemize}

            \item Все иные описания и свойства сущности следует признать 
            вторичными и зависящими от различных контекстов, а именно:
            \begin{itemize}
                \item связь сущностей между собой;
                \item атрибуты или свойства сущности.
            \end{itemize}

            \item Концепция придерживается 
            \href{https://ru.wikipedia.org/wiki/%D0%A0%D0%B0%D0%B7%D0%B4%D0%B5%D0%BB%D0%B5%D0%BD%D0%B8%D0%B5_%D0%BE%D1%82%D0%B2%D0%B5%D1%82%D1%81%D1%82%D0%B2%D0%B5%D0%BD%D0%BD%D0%BE%D1%81%D1%82%D0%B8}{принципа 
            разделения ответсвенности}, а потому отделяет описание сущностей и их 
            связей от реализации действий с ними и интерпритации в тех или иных 
            целях.

        \end{enumerate}


    \section{Практика}

        Для практических примеров применяется синтаксис YAML.

        \subsection{Первая сущность}

            \begin{enumerate}

                \item Для сущностей используется отношение кортежей в формате 
                ключ-значение, где ключ — это идентификатор сущности, а значение — 
                идентификатор её типа. 

                Это соответствует реляционному представлению с двумя атрибутами:

                \begin{itemize}
                    \item \texttt{id} - идентификатор сущности;
                    \item \texttt{type} - идентификатор типа сущности (который, в свою 
                    очередь является сущностью).
                \end{itemize}
                
                \item Поскольку тип сущности обязателен, а никакого типа еще не 
                существует, первую сущность следует самотипизировать во 
                множестве entities:

                \begin{verbatim}
                entities:
                    entity: "entity"
                \end{verbatim}


                \textit{Примечание:} Самотипизирующися кортеж создает новый домен. 
                Их количество не огранично, и каждый может развиваться независимо.

            \end{enumerate}

    \subsection{Развитие модели}

        \begin{enumerate}
            \item Определим несколько сущностей, которые в дальнейшем будут
            использоваться при описании архитектуры.

            \begin{verbatim}
            entities:

                # Базовые сущности архитектуры 

                # Компонент (сервисы, хранилища состояний и тд...)
                component: "entity"
                # Сервис как компонент
                service: "component"
                # База данных как компонент
                db: "component"
                # Агент обладающий возможностью действовать
                agent: "entity"
                # Определили пользователя, как агента
                user: "agent"
                # Определили клиента, как пользователя
                client: "user"

                # Добавим несколько специфичных для архитектуры сущностей:

                # Определим сервис backend
                my-backend: "service"
                # Определим базу данных 
                my-db: "db"
                # Определим внутреннего пользователя Алиса
                alice: "user"
                # Определим клиента Боб
                bob: "client"

            \end{verbatim}

            \item Таким образом последоватльными декларациями описаны 
            разнородные сущности, которые далее будут использованы.

        \end{enumerate}


    \subsection{Контексты}

        \begin{enumerate} 

            \item Описание сущности различается для участников с 
            учетом окружения. Такие различия определяются контекстом.

            \item Простейшим примером является описание сущности в 
            человекочитаемом виде на различных языках.

            \item Контекстом может быть не только язык, но любой специфический 
            взгляд на сущности со стороны различных групп, отделов, 
            подразделений, задач.

            \item Добавим несколько контекстов.

            \begin{verbatim}
            entities:

                # Контекст как сущность
                context: "entity"

                # Язык является конекстом
                lang: "contect"
                # Планы todo являются контекстом
                todo: "context"
                # Описание asis является контекстом
                asis: "context"

                # Русский язык
                ru: "lang"
                # Английский язык
                en: "lang"

            \end{verbatim}

        \end{enumerate}

    \subsection{Связи}
        \begin{enumerate}

            \item После добавления сущностей следует учитывать из взаимосвязи. 
            Для этого применяется секция links.

            \item Связи описываются как типизированное направления от одной сущности 
            к другой, при этом тип так же является сущностью. 

            \item Для связей возможно указание множества контекстов, которые 
            позволяют отобразить связи для различных ситуаций. Контекст является 
            опциональным. Отсутсвие указания контекста интерпритируется на 
            уровне представления.

            \item В общем виде связи могут быть определены кортежем атрибутов, 
            каждый из которых может содержать одну или более сущностей:

            \begin{itemize}
                \item \texttt{from} - сущность источник связи;
                \item \texttt{to} - сущность направление связи;
                \item \texttt{type} - сущность тип связи;
                \item \texttt{context} - опциональный список контекстов, для которых связь актуальна;
                \item \texttt{tuple} - опциональный список спекцифичных атрибутов 
                связи в формате ключ значение;
            \end{itemize}

            \item Опишем некоторые связи сущностей:

            \begin{verbatim}

            entities:
                # Связь как сущность
                link: "entity"

            # Определяем связи между сущностями в секции links
            links:         
                -              
                    # Определяем что сервис подключается а БД
                    from: "my-backend"
                    to: "my-db"
                    link: "connect"
                -
                    # Определяем что клиент подключается к сервису
                    from: "client"
                    to: "service"
                    link: "connect"
                -              
                    # Алиса может читать и добавлять данные
                    from: "alice"
                    link:
                        - "select"
                        - "insert"
                    to: "my-db"
                    context: "right"
                - 

                    # Боб обладает правом select для сервиса в контексте прав 
                    # для концепта asis
                    from: "bob" 
                    to: "service"
                    link: "select"
                    context: 
                        - "right"
                        - "asis"
                    # Планируется что Боб будет обладать правами создания и 
                    # добавления согласно концепта todo
                    from: "bob" 
                    to: "service"
                    link:
                        - "select"
                        - "insert"
                    context: 
                        - "right"
                        - "todo"

            \end{verbatim}

            \item Указанный способ описания может включать множество различных 
            зависимостей включая техническую связь компонентов, иерархические 
            структуры подчиненности, локацию размещений компонентов и прочее.
 
        \end{enumerate}


    \subsection{Описание сущностей}
        \begin{enumerate}

            \item Далее для сущностей возможно описание специфичных свойств
            так же в разрезе контекстов.

            \item В общем виде свойства сущностей моугт быть определены кортежем 
            атрибутов:

            \begin{itemize}
                \item \texttt{entity} - сущность для которой выполняется 
                описание, может содержать множество;
                \item \texttt{tuple} - список свойств сущности в 
                формате ключ значение;
                \item \texttt{context} - опциональный список контекстов, для 
                которых выполняется описание;
            \end{itemize}
        

            \item Описание сущностей производится в секции props:          
            \begin{verbatim}
            props:
                # Человекочитаемое описание для сущности
                -
                    entity: "entity"
                    context: "ru"
                    tuple: 
                        name:"Сущность"
                # Описание Алисы вне контекста
                -
                    entity: "alice"
                    tuple: 
                        age:21
                        weight:71
                # Описание свойств для Алисы и Боба на разных языках
                -
                    entity: "alice"
                    context: "ru"
                    tuple: 
                        name:"Алиса"
                -
                    entity: "alice"
                    context: "en"
                    tuple: 
                        name:"Alice"
                -
                    entity: "bob"
                    context: "ru"
                    tuple: 
                        name:"Боб"
                -
                    entity: "bob"
                    context: "en"
                    tuple: 
                        name:"Bob"
            
            \end{verbatim}

            \item Аналогичным образом возможно описание любых свойств сущностей.

        \end{enumerate}
        


    \section{Применимость}

        \begin{enumerate}

            \item Приведенные примеры показывают возможность использования 
            модели сущностей Catlair для описания моделей архитектуры.

            \item Нотация Catlair позволяет единообрзано описать перечнень 
            объектов, компонентов из взаимосвязи, при этом все перечисленное 
            может быть представлено с различных точек зрения, включая 
            хронологические.

            \item Модель применима для описания организационных структур, прав 
            доступа, технических компонентов.

            \item Модель может использоватся для единообразного формирования ER, 
            BPMN, C4 диаграмм на всех уровнях, при этом соблюдается единообразие 
            нотации, а формат представления определяется контекстом.

        \end{enumerate}
        

    \section{Приложение}

        Приложение содержит компактный листинг выше описанных примеров для 
        общего восприятия.

        \begin{verbatim}
        entities:
            entity: "entity"
            component: "entity"
            service: "component"
            db: "component"
            agent: "entity"
            user: "agent"
            client: "user"
            my-backend: "service"
            my-db: "db"
            alice: "user"
            bob: "client"
            link: "entity"
        links:         
            -              
                from: "my-backend"
                to: "my-db"
                link: "connect"
            -
                from: "client"
                to: "service"
                link: "connect"
            -              
                from: "alice"
                link:
                    - "select"
                    - "insert"
                to: "my-db"
                context: "right"
            - 
                from: "bob" 
                to: "service"
                link: "select"
                context: 
                    - "right"
                    - "asis"
                from: "bob" 
                to: "service"
                link:
                    - "select"
                    - "insert"
                context: 
                    - "right"
                    - "todo"
            props:
                -
                    entity: "entity"
                    context: "ru"
                    tuple: 
                        name:"Сущность"
                -
                    entity: "alice"
                    tuple: 
                        age:21
                        weight:71
                -
                    entity: "alice"
                    context: "ru"
                    tuple: 
                        name:"Алиса"
                -
                    entity: "alice"
                    context: "en"
                    tuple: 
                        name:"Alice"
                -
                    entity: "bob"
                    context: "ru"
                    tuple: 
                        name:"Боб"
                -
                    entity: "bob"
                    context: "en"
                    tuple: 
                        name:"Bob"
        \end{verbatim}

    \renewcommand{\refname}{Список литературы}
    \begin{thebibliography}{9}

        \bibitem{EWD447} EWD447
        \textit{On the role of scientific thought}
        \url{https://www.cs.utexas.edu/~EWD/transcriptions/EWD04xx/EWD447.html}

        \bibitem{SimonBrown}
        Simon Brown.\\
        \textit{Software Architecture as Code}.\\
        \url{https://static.codingthearchitecture.com/presentations/saturn2015-software-architecture-as-code.pdf}

        \bibitem{github}
        Cherkas R. Cheledinov I.\\
        \textit{Архитектура сущностей Catlair}.\\
        \url{https://github.com/johnthesmith/scraps/blob/main/ru/entities.md}

        \bibitem{RichardsFord}
        Mark Richards, Neal Ford.\\
        \textit{Is "Architecture as Code" the Future of Software Design?}\\
        \url{https://www.architectureandgovernance.com/applications-technology/is-architecture-as-code-the-future-of-software-design/}

    \end{thebibliography}

\end{document}


